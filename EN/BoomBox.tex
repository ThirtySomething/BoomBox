\documentclass[12pt,a4paper]{article}
\usepackage[T1]{fontenc}
\usepackage[utf8]{inputenc}
\usepackage[toc,page]{appendix}
\usepackage[english]{babel}
\usepackage{color}
\usepackage{etoolbox}
\usepackage{fancyhdr}
\usepackage{float}
\usepackage{graphicx}
\usepackage{makeidx}
\usepackage{imakeidx}
\usepackage{listings}
\usepackage{tocloft}

% \usepackage[margin=10pt,font=small,labelfont=bf,labelsep=endash]{caption}
% \usepackage[showframe]{geometry}
% \usepackage{afterpage}
% \usepackage{amsmath,amssymb}
% \usepackage{array}
% \usepackage{fncychap}
% \usepackage{hhline}
% \usepackage{latexsym}
% \usepackage{longtable}
% \usepackage{marvosym}
% \usepackage{pdflscape}
% \usepackage{setspace}
% \usepackage{stmaryrd}
% \usepackage{tabularx}
% \usepackage{wasysym}

% Definition of page style with fancy header
\fancypagestyle{lscape}{
\fancyhf{}
\fancyfoot[LE]{
\begin{textblock}{20} (1,5){\rotatebox{90}{\leftmark}}\end{textblock}
\begin{textblock}{1} (13,10.5){\rotatebox{90}{\thepage}}\end{textblock}}
\fancyfoot[LO] {
\begin{textblock}{1} (13,10.5){\rotatebox{90}{\thepage}}\end{textblock}
\begin{textblock}{20} (1,13.25){\rotatebox{90}{\rightmark}}\end{textblock}}
\renewcommand{\headrulewidth}{0pt}
\renewcommand{\footrulewidth}{0pt}}

% Settings for dotted line in table of content
\pagestyle{fancy}
\renewcommand{\cftpartleader}{\cftdotfill{\cftdotsep}}
\renewcommand{\cftsecleader}{\cftdotfill{\cftdotsep}}
\setlength{\headheight}{15pt}

% \makeatletter
% \def\verbatim{
%     \scriptsize,
%     \@verbatim,
%     \frenchspacing,
%     \@vobeyspaces,
%     \@xverbatim,
% }
% \makeatother

% \makeatletter
% \preto{\@verbatim}{\topsep=0pt \partopsep=0pt }
% \makeatother

\author{ThirtySomething}
\title{BoomBox}
\date{\today}

\usepackage[
pdftex,
pdfauthor={ThirtySomething -- https://www.derpaul.net},
pdftitle={BoomBox},
pdfsubject={A portable music player},
colorlinks=true,
linkcolor=blue,
urlcolor=blue
]{hyperref}
\usepackage[all]{hypcap}

\pagestyle{fancy}
\fancyhf{}
\renewcommand{\sectionmark}[1]{\markright{#1}}
\renewcommand{\subsectionmark}[1]{\markright{#1}}
\renewcommand{\subsubsectionmark}[1]{\markright{#1}}
\fancyhead[R]{BoomBox}
\fancyhead[L]{\nouppercase{\rightmark}}
\fancyfoot[C]{Page \thepage}
\renewcommand{\headrulewidth}{0.4pt}
\renewcommand{\footrulewidth}{0.4pt}

\def\labelitemi{--}
\newcommand{\bb}{\textit{\href{https://github.com/ThirtySomething/BoomBox}{BoomBox}}}
\newcommand{\code}[1]{\texttt{#1}}
\newcommand{\jpaimg}[2]{\begin{figure}[H]\centering\fbox{\includegraphics[width=380px]{#1}}\caption{#2}\label{fig:#2}\end{figure}}
\newcommand{\rpi}{\href{https://www.raspberrypi.org/}{Raspberry Pi}\index{Raspberry Pi}}
\newcommand{\vol}{\href{https://volumio.org/}{Volumio}\index{Volumio}}

% Settings for bash commands (lstlisting)
\definecolor{dkgreen}{rgb}{0,0.6,0}
\definecolor{gray}{rgb}{0.5,0.5,0.5}
\definecolor{mauve}{rgb}{0.58,0,0.82}

\lstset{frame=tb,
  language=sh,
  aboveskip=3mm,
  belowskip=3mm,
  showstringspaces=false,
  columns=flexible,
  basicstyle={\scriptsize\ttfamily},
  numbers=none,
  numberstyle=\tiny\color{gray},
  keywordstyle=\color{blue},
  commentstyle=\color{dkgreen},
  stringstyle=\color{mauve},
  breaklines=true,
  breakatwhitespace=true,
  tabsize=3
}

\makeindex

\begin{document}
\clearpage\maketitle
\thispagestyle{empty}
\newpage

\tableofcontents
\addtocontents{toc}{\protect\thispagestyle{fancy}}
\newpage

\section{Motivation}
There are many ways for parties to play music. One of the modern variants of this is streaming via a smartphone. However, this requires a functioning WLAN or
mobile phone reception. If you want to use the telephone network, you need a corresponding data connection or the corresponding data volume. But what do you do
if neither one nor the other is given? Then you use the \bb{}. This is a kind of modern ghetto blaster. The device is independent of smartphone and WLan or
mobile phone reception.

\section{The device}
The device \bb{} consists of two components, the hardware component and the software component.

\subsection{The hardware component}
On the one hand, the system should be cost-effective, but on the other hand it should not be old-fashioned. The following components meet these requirements:

\begin{itemize}
    \item A \href{https://www.raspberrypi.org/products/raspberry-pi-3-model-b-plus/}{\rpi{} 3 B+}.
    \item An \href{http://iqaudio.co.uk/hats/9-pi-digiamp.html}{Pi-DigiAMP+}\index{iQAudio}\index{Pi-DigiAMP+}.
    \item A \href{https://www.raspberrypi.org/products/raspberry-pi-touch-display/}{\rpi{} Touch Display}\index{Touch Display}.
    \item A \href{https://www.conrad.de/de/m2-sata-ssd-erweiterungs-platine-fuer-den-raspberry-pi-1487097.html}{M.2 to USB Adapter}\index{M.2}.
    \item A \href{https://www.wd.com/de-de/products/internal-ssd/wd-blue-3d-nand-sata-ssd.html}{250 GB M.2 SSD}\index{SSD}.
    \item A \href{https://www.amazon.de/gp/product/B002JIGJ4M/ref=ppx_yo_dt_b_asin_title_o04_s00?ie=UTF8&psc=1}{Power supply}.
    \item Two \href{https://www.amazon.de/gp/product/B01GJC4WRO/ref=ppx_yo_dt_b_asin_title_o07_s00?ie=UTF8&psc=1}{USB cable}.
    \item A \href{https://www.amazon.de/gp/product/B073S9SFK2/ref=ppx_yo_dt_b_asin_title_o07_s00?ie=UTF8&psc=1}{MicroSD card}.
    \item A \href{https://www.amazon.de/gp/product/B07KFFNBLJ/ref=ppx_yo_dt_b_asin_title_o03_s00?ie=UTF8&psc=1}{Step Down Converter}\index{Step Down Converter}.
    \item An \href{https://www.amazon.de/gp/product/B071KVWQKY/ref=ppx_yo_dt_b_asin_title_o05_s00?ie=UTF8&psc=1}{Adatper with terminal block}.
    \item A \href{https://www.amazon.de/gp/product/B00A6QKIEQ/ref=ppx_yo_dt_b_asin_title_o06_s00?ie=UTF8&psc=1}{Cable with plug}.
\end{itemize}

The \rpi{} is an inexpensive single board computer that is perfect for this project. The sound card greatly improves audio output -- the \rpi{} isn't so
convincing here by nature. The original touch display is used for operation without additional input devices. On Parties can be a bit wild every now and then.
So that shocks don't have any influence on the \bb{}, instead of a conventional hard disk an SSD is used as mass storage. This is connected to the system via
an adapter.

\subsection{The software component}
The software essentially consists of only one point: the music distibution \vol{}. This distribution comes with support for the above hardware. Of course
some fine-tuning is necessary to simplify operation and handling. This is described in the chapter~\ref{subsec:Fine-tuning}.

\section{Preparations}
Before the entire system can go to the start, a few preparations have to be made.

\subsection{Updating the \rpi{} firmware}
We'll connect the \rpi{} with a network cable. Then we download the latest version of \href{https://www.raspberrypi.org/downloads/raspbian/}{Raspbian}. This
is the normal operating system for the \rpi{}. We write the image with the \href{https://sourceforge.net/projects/win32diskimager/}{Win32 Disk Imager} on
a MicroSD card. To allow access via SSH, we create the empty file \textit{ssh}\index{ssh} in the boot partition. We boot the \rpi{} and connect to
\textit{ssh}. Then we enter the following commands:

\begin{figure}[H]
\begin{lstlisting}
sudo apt-get update
sudo apt-get install git
sudo wget https://raw.github.com/Hexxeh/rpi-update/master/rpi-update -O /usr/bin/rpi-update && sudo chmod +x /usr/bin/rpi-update
sudo rpi-update
sudo reboot
\end{lstlisting}
\caption{Firmware Update}\label{fig:Firmware Update}
\end{figure}

\subsection{SSD mounting}
The SSD must be mounted on the adapter. This is very simple - insert the SSD into the slot, tighten the screw and you're done. The result looks like this:

\jpaimg{./../images/ssd-prepared.png}{SSD with adapter}


\subsection{Prepare SSD}
We format the SSD with \href{https://en.wikipedia.org/wiki/Ext4}{ext4}. This means that the SSD can no longer be used directly on the Windows PC.\@ However,
the file system is more robust than \href{https://en.wikipedia.org/wiki/File_Allocation_Table#FAT32}{FAT32}. This is especially true in the event of sudden
power loss. If you later include the \bb{} in your own network, you get access to the SSD.\@

\subsection{Filling up the SSD}\label{subsec:Filling up the SSD}
To be able to offer \vol{} also a music selection at the start, the SSD is initially refuelled. For this a SMB-Share is connected and the files are copied.
This step must only be performed once.

\begin{figure}[H]
\begin{lstlisting}
# Utilities to mount the SMB-Shares
sudo apt-get install cifs-utils
# Mount SMB-Share
mount -t cifs -o user=<smbuser>,domain=<domain|workgroup> //<IP of the share>/<sharename> /mnt
# Create mountpoint /music for the SSD
sudo mkdir /music
# Mount the SSD to /music
sudo mount /dev/sda1 /music
# Start filling up of the SSD
sudo cd /mnt
sudo cp -R * /music
\end{lstlisting}
\caption{Filling up the SSD}\label{fig:Filling up the SSD}
\end{figure}

After the copy process is finished, the \rpi{} can be shut down. We remove the MicroSD card. For the next use we save \vol{} on it.

\section{The hardware component}
This is about the mechanical assembly of the \bb{}. Since everything is stacked on top of each other, I'm also talking about the \textit{Hardwarestack}.

\subsection{The display and the \rpi{}}
We'll start with the display. When unpacking, it is noticeable that the control board is already connected and mounted.

\jpaimg{./../images/display.png}{Display}

This simplifies assembly for us. How this is done is simply explained in this \href{https://www.youtube.com/watch?v=tK-w-wDvRTg}{YouTube Video}.

\textbf{Caution:} For the \bb{} we only connect the ribbon cable. In the video, the \rpi{} is fastened with screws. Instead of these screws we use spacer
bolts M2,5~x~11mm. After the \rpi{} the sound card and the converter board for the SSD will be added.

\jpaimg{./../images/dsp-pi.png}{Display with Pi}

\subsection{The sound card}
There's not much to explain here. The sound card is placed on the GPIO bar of the \rpi{}. Then the spacer bolts, which were supplied with the SSD adapter
board (!), are screwed on for fixing.

\jpaimg{./../images/dsp-pi-iq.png}{Display, Pi and iQAudio}

\subsection{The converter board}
At the end comes the SSD with the converter board. We have already connected both in the preparations. This board is fixed with screws on the spacer bolts of
the sound card. The \rpi{} is powered by the soundcard. However, this is too less to supply the converter board with the SSD via USB.\@ Therefore we have to set
the jumper \textit{PWR\_U} so that the middle pin and the pin closest to the board edge are bridged. This ensures that the converter board is not supplied with
power via USB, but via the extra input.

\jpaimg{./../images/dsp-pi-iq-ssd.png}{Display, Pi, iQAudio and SSD}

\subsection{Adapter cable}
The power supply has only one output with 19V. For the display and the converter board 5V are required. For this we need an adapter cable. The cable has a
socket, into which the plug of the power supply comes. This socket has screw terminals on the other side. We connect two cables to these screw terminals. One,
which again has a plug for the sound card. And one which is connected with screw terminals to a so-called
\href{https://en.wikipedia.org/wiki/Buck_converter}{Step Down Converter}. This Step Down Converter has two USB ports, which we use to power the display
and the converter board.

\jpaimg{./../images/adaptercable.png}{Adapter cable}

\subsection{The result}
If everything was assembled correctly, it looks like in the following picture.

\jpaimg{./../images/bbwopwr.png}{Hardwarestack}

And if the cables were also connected, it looks that way:

\jpaimg{./../images/cableconnected.png}{Hardwarestack with cables}

\section{The software component}
This is about the installation and configuration of \vol{}.

\subsection{First installation}
Prerequisites for the installation is the \nameref{subsec:Filling up the SSD}. And of course the assembly of the hardware stack. For this we download the
image of \vol{}. After that we try the Win32 Disk Imager again and save the image on the MicroSD card. After the image is installed, the card is inserted in
the \rpi{}, we start the system. Please make sure that the \rpi{} is connected to the network with a network cable.

\subsection{The plugins}
We can find out the IP address of the \bb{} via our router. Then we call up the IP address in the browser. The start screen will look like this.

\jpaimg{./../images/vol-main.png}{Initial screen}

By the time we see this image, we have already made a great deal of progress. In order for the touch screen to work, an appropriate plugin is required. To
do this we go through the settings -- the gear in the upper left corner.

\jpaimg{./../images/vol-setup.png}{Settings}

For the plugins we select the corresponding menu item. The plugin for the touchscreen can be found under \textit{Miscellanea}, it is called
\textit{Touch Display Plugin}.

\jpaimg{./../images/vol-touch.png}{Touchscreen}

Another plugin is a simple equalizer. We install it as well. It can be found under \textit{Audio Interface}.

\jpaimg{./../images/vol-equal.png}{Equalizer}

After the plugins have been installed, you have to activate them. This is done on the second tab \textit{Installed Plugins}. After they are activated, it
looks like this.

\jpaimg{./../images/vol-plug-active.png}{Plugins}

\subsection{Fine-tuning}\label{subsec:Fine-tuning}
Now it's time for some fine tuning. How to do them is explained
\href{https://volumio.org/forum/guide-for-setting-touchscreen-backlight-control-t11425.html}{on this page}.

\textbf{Note:} After performing one or more of these configuration steps, a reboot is necessary. Only then the changes will take effect.

\subsubsection{The mouse pointer}\label{subsubsec:SSH}
We start by hiding the mouse pointer. First we have to activate SSH.\@ This can be done via the browser with the following URL:\@ \\
\textit{http://<IP-the-BoomBox>/dev} -- in my case for example with \\ \textit{http://192.168.2.17/dev}. On this page we find buttons to activate and
de\-activate access with SSH.\@ For our purpose we need this actively.

\jpaimg{./../images/vol-dev.png}{SSH}

Then we log on to the system via ssh. The username is \textit{volumio}; the password is identical to the username. Then we edit the file configuration file for
the kiosk mode. We add \code{-{}- -nocursor} to the line.

\begin{figure}[H]
\begin{lstlisting}
sudo nano /lib/systemd/system/volumio-kiosk.service
# Original line
# ExecStart=/usr/bin/startx /etc/X11/Xsession /opt/volumiokiosk.sh
# Modified line
ExecStart=/usr/bin/startx /etc/X11/Xsession /opt/volumiokiosk.sh -- -nocursor
# Leave the editor by pressing CTRL+X
\end{lstlisting}
\caption{Kiosk mode}\label{fig:Kiosk mode}
\end{figure}

\subsubsection{Screensaver}
From time to time the display is simply \textit{switched off}. That is, it becomes completely black. To prevent this, the following steps are necessary:

\begin{figure}[H]
\begin{lstlisting}
sudo nano /opt/volumiokiosk.sh

# Original lines
# xset +dpms
# xset s blank
# xset 0 0 120

# Adjusted lines
xset -dpms
xset s off
#xset 0 0 120
# Leave the editor by pressing CTRL+X
\end{lstlisting}
\caption{Screensaver}\label{fig:Screensaver}
\end{figure}

\subsubsection{Access from Windows}
At \vol{} a samba is already installed by default. This allows easy access to the storage. However, the device offers different storage locations. This could
cause confusion. That's why we make sure that only storage that is connected via USB can be accessed. So we can access the SSD from Windows without guesswork.

\begin{figure}[H]
\begin{lstlisting}
sudo nano /etc/samba/smb.conf

# Original lines
[Internal Storage]
        comment = Boombox Internal Music Folder
        path = /data/INTERNAL
        read only = no
        guest ok = yes

[USB]
        comment = Boombox USB Music Folder
        path = /mnt/USB
        read only = no
        guest ok = yes

[NAS]
        comment = Boombox NAS Music Folder
        path = /mnt/NAS
        read only = no
        guest ok = yes

# Adjusted lines
#[Internal Storage]
#        comment = Boombox Internal Music Folder
#        path = /data/INTERNAL
#        read only = no
#        guest ok = yes

[SSD]
        comment = Boombox SSD Music Folder
        path = /mnt/USB
        read only = no
        guest ok = yes

#[NAS]
#        comment = Boombox NAS Music Folder
#        path = /mnt/NAS
#        read only = no
#        guest ok = yes

# Leave the editor by pressing CTRL+X
\end{lstlisting}
\caption{Share}\label{fig:Share}
\end{figure}

Under Windows, the device can then be accessed under the name \code{\textbackslash{}\textbackslash{}boombox} in Windows Explorer.

\jpaimg{./../images/win-bb.png}{Access from Windows}

Finally, we turn on the \textit{SSH} access again. To do this we call up the corresponding page. See also chapter \nameref{subsubsec:SSH}.

\clearpage{}
\phantomsection{}
\addcontentsline{toc}{section}{List of figures}
\listoffigures\thispagestyle{fancy}
\newpage

% \clearpage{}
% \phantomsection{}
% \addcontentsline{toc}{section}{Tabellenverzeichnis}
% \listoftables\thispagestyle{fancy}
% \newpage

\renewcommand{\indexname}{Index}
\clearpage{}
\phantomsection{}
\addcontentsline{toc}{section}{Index}
\printindex\thispagestyle{fancy}
\newpage

\end{document}
