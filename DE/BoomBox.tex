\documentclass[12pt,a4paper]{article}
\usepackage[T1]{fontenc}
\usepackage[utf8]{inputenc}
\usepackage[toc,page]{appendix}
\usepackage[ngerman]{babel}
\usepackage{color}
\usepackage{etoolbox}
\usepackage{fancyhdr}
\usepackage{float}
\usepackage{graphicx}
\usepackage{makeidx}
\usepackage{imakeidx}
\usepackage{listings}
\usepackage{tocloft}

% \usepackage[margin=10pt,font=small,labelfont=bf,labelsep=endash]{caption}
% \usepackage[showframe]{geometry}
% \usepackage{afterpage}
% \usepackage{amsmath,amssymb}
% \usepackage{array}
% \usepackage{fncychap}
% \usepackage{hhline}
% \usepackage{latexsym}
% \usepackage{longtable}
% \usepackage{marvosym}
% \usepackage{pdflscape}
% \usepackage{setspace}
% \usepackage{stmaryrd}
% \usepackage{tabularx}
% \usepackage{wasysym}

% Definition of page style with fancy header
\fancypagestyle{lscape}{
\fancyhf{}
\fancyfoot[LE]{
\begin{textblock}{20} (1,5){\rotatebox{90}{\leftmark}}\end{textblock}
\begin{textblock}{1} (13,10.5){\rotatebox{90}{\thepage}}\end{textblock}}
\fancyfoot[LO] {
\begin{textblock}{1} (13,10.5){\rotatebox{90}{\thepage}}\end{textblock}
\begin{textblock}{20} (1,13.25){\rotatebox{90}{\rightmark}}\end{textblock}}
\renewcommand{\headrulewidth}{0pt}
\renewcommand{\footrulewidth}{0pt}}

% Settings for dotted line in table of content
\pagestyle{fancy}
\renewcommand{\cftpartleader}{\cftdotfill{\cftdotsep}}
\renewcommand{\cftsecleader}{\cftdotfill{\cftdotsep}}
\setlength{\headheight}{15pt}

% \makeatletter
% \def\verbatim{
%     \scriptsize,
%     \@verbatim,
%     \frenchspacing,
%     \@vobeyspaces,
%     \@xverbatim,
% }
% \makeatother

% \makeatletter
% \preto{\@verbatim}{\topsep=0pt \partopsep=0pt }
% \makeatother

\author{ThirtySomething}
\title{BoomBox}
\date{\today}

\usepackage[
pdftex,
pdfauthor={ThirtySomething -- https://www.derpaul.net},
pdftitle={BoomBox},
pdfsubject={Ein portables Musikabspielgerät},
colorlinks=true,
linkcolor=blue,
urlcolor=blue
]{hyperref}
\usepackage[all]{hypcap}

\pagestyle{fancy}
\fancyhf{}
\renewcommand{\sectionmark}[1]{\markright{#1}}
\renewcommand{\subsectionmark}[1]{\markright{#1}}
\renewcommand{\subsubsectionmark}[1]{\markright{#1}}
\fancyhead[R]{BoomBox}
\fancyhead[L]{\nouppercase{\rightmark}}
\fancyfoot[C]{Seite \thepage}
\renewcommand{\headrulewidth}{0.4pt}
\renewcommand{\footrulewidth}{0.4pt}

\def\labelitemi{--}
\newcommand{\bb}{\textit{\href{https://github.com/ThirtySomething/BoomBox}{BoomBox}}}
\newcommand{\code}[1]{\texttt{#1}}
\newcommand{\jpaimg}[2]{\begin{figure}[H]\centering\fbox{\includegraphics[width=380px]{#1}}\caption{#2}\label{fig:#2}\end{figure}}
\newcommand{\rpi}{\href{https://www.raspberrypi.org/}{Raspberry Pi}\index{Raspberry Pi}}
\newcommand{\vol}{\href{https://volumio.org/}{Volumio}\index{Volumio}}

% Settings for bash commands (lstlisting)
\definecolor{dkgreen}{rgb}{0,0.6,0}
\definecolor{gray}{rgb}{0.5,0.5,0.5}
\definecolor{mauve}{rgb}{0.58,0,0.82}

\lstset{frame=tb,
  language=sh,
  aboveskip=3mm,
  belowskip=3mm,
  showstringspaces=false,
  columns=flexible,
  basicstyle={\scriptsize\ttfamily},
  numbers=none,
  numberstyle=\tiny\color{gray},
  keywordstyle=\color{blue},
  commentstyle=\color{dkgreen},
  stringstyle=\color{mauve},
  breaklines=true,
  breakatwhitespace=true,
  tabsize=3
}

\makeindex

\begin{document}
\clearpage\maketitle
\thispagestyle{empty}
\newpage

\tableofcontents
\addtocontents{toc}{\protect\thispagestyle{fancy}}
\newpage

\section{Motivation}
Es gibt für Parties viele Möglichkeiten, Musik abzuspielen. Eine der modernen Varianten davon ist Streaming über ein Smartphone. Allerdings setzt das
ein funktionierendes WLAN bzw. Handyempfang voraus. Sofern es über das Telefonnetz gehen soll, ist natürlich eine entsprechende Datenverbindung bzw.~das
dazugehörige Datenvolumen Voraussetzung. Was macht man jedoch, wenn weder das eine noch das andere gegeben ist? Dann greift man zur \bb{}.
Das ist eine Art moderner Ghettoblaster. Das Gerät ist unabhängig von Smartphone und WLan bzw. Handyempfang.

\section{Das Gerät}
Das Gerät \bb{} besteht aus zwei Komponenten, der Hardwarekomponente und der Softwarekomponente.

\subsection{Die Hardwarekomponente}
Das System soll zum einen kostengünstig, zum anderen aber auch nicht altbacken sein. Folgende Komponenten erfüllen diese Anforderungen:

\begin{itemize}
    \item Einem \href{https://www.raspberrypi.org/products/raspberry-pi-3-model-b-plus/}{\rpi{} 3 B+}.
    \item Einem \href{http://iqaudio.co.uk/hats/9-pi-digiamp.html}{Pi-DigiAMP+}\index{iQAudio}\index{Pi-DigiAMP+}.
    \item Einem \href{https://www.raspberrypi.org/products/raspberry-pi-touch-display/}{\rpi{} Touch Display}\index{Touch Display}.
    \item Einem \href{https://www.conrad.de/de/m2-sata-ssd-erweiterungs-platine-fuer-den-raspberry-pi-1487097.html}{M.2 zu USB Adapter}\index{M.2}.
    \item Einer \href{https://www.wd.com/de-de/products/internal-ssd/wd-blue-3d-nand-sata-ssd.html}{250 GB M.2 SSD}\index{SSD}.
    \item Einem \href{https://www.amazon.de/gp/product/B002JIGJ4M/ref=ppx_yo_dt_b_asin_title_o04_s00?ie=UTF8&psc=1}{Netzteil}.
    \item Zwei \href{https://www.amazon.de/gp/product/B01GJC4WRO/ref=ppx_yo_dt_b_asin_title_o07_s00?ie=UTF8&psc=1}{USB Kabel}.
    \item Einer \href{https://www.amazon.de/gp/product/B073S9SFK2/ref=ppx_yo_dt_b_asin_title_o07_s00?ie=UTF8&psc=1}{MicroSD Karte}.
    \item Einem \href{https://www.amazon.de/gp/product/B07KFFNBLJ/ref=ppx_yo_dt_b_asin_title_o03_s00?ie=UTF8&psc=1}{Step Down Converter}\index{Step Down Converter}.
    \item Einem \href{https://www.amazon.de/gp/product/B071KVWQKY/ref=ppx_yo_dt_b_asin_title_o05_s00?ie=UTF8&psc=1}{Adatper mit Terminalblock}.
    \item Einem \href{https://www.amazon.de/gp/product/B00A6QKIEQ/ref=ppx_yo_dt_b_asin_title_o06_s00?ie=UTF8&psc=1}{Kabel mit Stecker}.
\end{itemize}

Der \rpi{} ist ein günstiger Einplatinencomputer, der für dieses Projekt wie geschaffen ist. Die Soundkarte verbessert die Audioausgabe erheblich -- der \rpi{}
ist von Haus aus hier nicht so überzeugend. Damit eine Bedienung ohne zusätzliche Eingabegeräte funktioniert, wird das Original Touch Display verwendet. Auf
Parties kann es ja ab und zu mal etwas wilder zugehen. Damit Erschütterungen keinen Einfluss auf die \bb{} haben, wird statt einer konventionellen Festplatte
eine SSD als Massenspeicher eingesetzt. Diese wird über einen Adapter an das System angeschlossen.

\subsection{Die Softwarekomponente}
Die Software besteht im Wesentlichen aus nur einem Punkt: Der Musikdistibution \vol{}. Diese Distribution bringt von Haus aus bereits eine Unterstützung für
obige Hardware mit. Natürlich ist etwas Feintuning notwendig, um die Bedienung und das Handling zu vereinfachen. Das wird im Kapitel~\ref{subsec:Feintuning}
beschrieben.

\section{Vorbereitungen}
Bevor das Gesamtsystem an den Start gehen kann, sind noch ein paar Vorbereitungen zu treffen.

\subsection{Firmware des \rpi{} aktualisieren}
Den \rpi{} schliessen wir mit einem Netzwerkkabel an. Dann laden wir die aktuellste Version von \href{https://www.raspberrypi.org/downloads/raspbian/}{Raspbian}
herunter. Das ist das normale Betriebssystem für den \rpi{}. Wir schreiben das Image mit dem \href{https://sourceforge.net/projects/win32diskimager/}{Win32 Disk
Imager} auf eine MicroSD Karte. Um den Zugriff via SSH zu ermöglichen, erzeugen wir in der Boot Partition noch die leere Datei \textit{ssh}\index{ssh}. Wir
booten den \rpi{} und verbinden uns mit \textit{ssh}. Dann geben wir folgende Kommandos ein:

\begin{figure}[H]
\begin{lstlisting}
sudo apt-get update
sudo apt-get install git
sudo wget https://raw.github.com/Hexxeh/rpi-update/master/rpi-update -O /usr/bin/rpi-update && sudo chmod +x /usr/bin/rpi-update
sudo rpi-update
sudo reboot
\end{lstlisting}
\caption{Firmware Upgrade}\label{fig:Firmware Upgrade}
\end{figure}

\subsection{SSD Montage}
Die SSD muss auf den Adapter montiert werden. Das ist ganz einfach - SSD in den Slot einstecken, die Schraube zudrehen, fertig. Das Ergebnis sieht dann so aus:

\jpaimg{./../images/ssd-prepared.png}{SSD mit Adapter}


\subsection{SSD Vorbereiten}
Die SSD formatieren wir mit \href{https://de.wikipedia.org/wiki/Ext4}{ext4}. Damit kann die SSD nicht mehr direkt am Windows PC verwendet werden. Allerdings
ist das Dateisystem robuster als \href{https://de.wikipedia.org/wiki/File_Allocation_Table#FAT32}{FAT32}. Das gilt besonders in Hinblick auf einen plötzlichen
Stromverlust. Sofern man später die \bb{} in das eigene Netzwerk aufnimmt, erhält man über einen Samba Share Zugriff auf die SSD.\@

\subsection{Betanken der SSD}\label{subsec:Betanken der SSD}
Um \vol{} beim Start auch eine Musikauswahl anbieten zu können, wird die SSD initial betankt. Hierzu wird ein SMB-Share verbunden und die Dateien
kopiert. Dieser Schritt ist nur einmalig durchzuführen.

\begin{figure}[H]
\begin{lstlisting}
# Utilities zum Mounten des SMB-Shares
sudo apt-get install cifs-utils
# SMB-Share mounten
mount -t cifs -o user=<smbuser>,domain=<domain|workgroup> //<IP des Shares>/<sharename> /mnt
# Mountpoint /music fuer SSD anlegen
sudo mkdir /music
# SSD nach /music mounten
sudo mount /dev/sda1 /music
# Betankung starten
sudo cd /mnt
sudo cp -R * /music
\end{lstlisting}
\caption{SSD Betanken}\label{fig:SSD Betanken}
\end{figure}

Nachdem der Kopiervorgang beendet ist, kann der \rpi{} heruntergefahren werden. Die MicroSD Karte entfernen wir. Für die nächste Verwendung spielen wir da
\vol{} auf.

\section{Die Hardwarekomponente}
Hier geht es um den mechanischen Zusammenbau der \bb{}. Da alles übereinander gestapelt wird, spreche ich auch von dem \textit{Hardwarestack}.

\subsection{Das Display und der \rpi{}}
Wir beginnen mit dem Display. Beim Auspacken fällt auf, dass die Steuerplatine bereits angeschlossen und montiert ist.

\jpaimg{./../images/display.png}{Display}

Das vereinfacht den Zusammenbau für uns. Wie das genau gemacht wird, ist in diesem \href{https://www.youtube.com/watch?v=tK-w-wDvRTg}{YouTube Video} einfach
erklärt. \textbf{Achtung:} Für die \bb{} schliessen wir nur das Flachbandkabel an. Im Video wird der \rpi{} mit Schrauben befestigt. Statt diesen Schrauben
verwenden wir Abstandsbolzen M2,5~x~11mm. Nach dem \rpi{} kommen ja noch die Soundkarte und die Konverterplatine.

\jpaimg{./../images/dsp-pi.png}{Display mit Pi}


\subsection{Die Soundkarte}
Hier gibt es nicht viel zu erklären. Die Soundkarte wird auf die GPIO-Leiste des \rpi{} gesteckt. Danach werden die Abstandbolzen, die bei der
SSD-Adapterplatine (!) mitgeliefert wurden, zur Fixierung aufgeschraubt.

\jpaimg{./../images/dsp-pi-iq.png}{Display, Pi und iQAudio}

\subsection{Die Konverterplatine}
Zum Schluss kommt die SSD mit der Konverterplatine. Beide haben wir ja bereits in den Vorbereitungen miteinander verbunden. Diese Platine wird mit Schrauben auf
den Abstandsbolzen der Soundkarte fixiert. Der \rpi{} wird zwar über die Soundkarte mit Strom versorgt. Das ist allerdings zu wenig, um auch noch über USB die
Konverterplatine mit der SSD zu versorgen. Deswegen müssen wir den Jumper \textit{PWR\_U} so setzen, dass der mittlere Pin und der dem Platinenrand am nächsten
stehende Pin gebrückt sind. Das sorgt dafür, dass die Konverterplatine nicht über USB, sondern über den extra Eingang mit Strom versorgt wird.

\jpaimg{./../images/dsp-pi-iq-ssd.png}{Display, Pi, iQAudio und SSD}

\subsection{Adapterkabel}
Das Netzteil hat nur einen Ausgang mit 19V. Für das Display und die Konverterplatine werden jedoch 5V benötigt. Dazu benötigen wir ein Adapterkabel. Das Kabel
hat eine Buchse, in welche der Stecker des Netzteils kommt. Diese Buchse hat auf der anderen Seite Schraubklemmen. An diese Schraubklemmen schliessen wir zwei
Kabel an. Eines, welches wieder einen Stecker für die Soundkarte hat. Und eines, welches mit Schraubklemmen an einem sogenannten
\href{https://de.wikipedia.org/wiki/Abw\%C3\%A4rtswandler}{Step Down Converter} verbunden wird. Dieser Step Down Converter hat zwei USB Anschlüsse, über die wir
dann das Display und die Konverterplatine mit Strom versorgen.

\jpaimg{./../images/adaptercable.png}{Adapterkabel}

\subsection{Das Ergebnis}
Wenn alles korrekt zusammengebaut wurde, sieht es dann aus wie in folgendem Bild.

\jpaimg{./../images/bbwopwr.png}{Hardwarestack}

Und wenn dann noch die Kabel angeschlossen wurden, sieht es so aus:

\jpaimg{./../images/cableconnected.png}{Hardwarestack mit Kabel}

\section{Die Softwarekomponente}
Hier geht es um die Installation und Konfiguration von \vol{}.

\subsection{Erstinstallation}
Voraussetungen für die Installation ist das \nameref{subsec:Betanken der SSD}. Und natürlich der Zusammenbau des Hardwarestacks. Dazu laden wir das Image von
\vol{} herunter. Danach bemühen wir wieder den Win32 Disk Imager und spielen damit das Image auf die MicroSD Karte. Nachdem das Image aufgespielt ist, die Karte
im \rpi{} eingesetzt wurde, starten wir das System. Bitte darauf achten, das der \rpi{} mit einem Netzwerkkabel an das Netzwerk angeschlossen ist.

\subsection{Die Plugins}
Die IP Adresse der \bb{} können wir über unseren Router herausfinden. Danach rufen wir im Browser die IP Adresse auf. Der Startbildschirm sieht dann in etwa
so aus.

\jpaimg{./../images/vol-main.png}{Startbildschirm}

Wenn wir diesen Bildschim sehen, haben wir schon einen sehr großen Fortschritt erzielt. Damit der Touchscreen funktioniert, ist ein entsprechendes Plugin
notwendig. Hierzu gehen wir über die Einstellungen -- das Zahnrad in der linken, oberen Ecke.

\jpaimg{./../images/vol-setup.png}{Einstellungen}

Für die Plugins wählen wir den entsprechenden Menüpunkt aus. Das Plugin für den Touchscreen finden wir unter \textit{Miscellanea}, es heisst \textit{Touch
Display Plugin}.

\jpaimg{./../images/vol-touch.png}{Touchscreen}

Ein weiteres Plugin ist ein einfacher Equalizer. Diesen installieren wir ebenso. Er ist unter \textit{Audio Interface} zu finden.

\jpaimg{./../images/vol-equal.png}{Equalizer}

Nachdem die Plugins installiert wurden, muss man sie natürlich noch aktivieren. Das geht auf dem zweiten Reiter \textit{Installierte Plugins}. Nachdem sie
aktiviert wurden, sieht das aus wie folgt.

\jpaimg{./../images/vol-plug-active.png}{Plugins}

\subsection{Feintuning}\label{subsec:Feintuning}
Jetzt geht es an ein paar Feineinstellungen. Wie die zu machen sind, ist \href{https://volumio.org/forum/guide-for-setting-touchscreen-backlight-control-t11425.html}{auf dieser Seite}
erklärt. \textbf{Hinweis:} Nachdem einer oder mehrerer dieser Konfigurationsschritte durchgeführt wurde, ist ein Reboot notwendig. Erst damit werden die
Änderungen wirksam.

\subsubsection{Der Mauszeiger}\label{subsubsec:SSH}
Wir beginnen mit dem Ausblenden des Mauszeigers. Zuerst müssen wir hierfür SSH aktivieren. Das geht über den Browser mit folgender URL:\@ \\
\textit{http://<IP-der-BoomBox>/dev} -- in meinem Fall beispielsweise mit \\ \textit{http://192.168.2.17/dev}. Auf dieser Seite finden wir Buttons, über die der
Zugang mit SSH aktiviert und deaktivert werden kann. Für unseren Zweck brauchen wir das aktiv.

\jpaimg{./../images/vol-dev.png}{SSH}

Danach melden wir uns über ssh an dem System an. Der Benutzername ist \textit{volumio}; das Passwort ist identisch zum Benutzernamen. Dann editieren wir die
Datei Konfigurationsdatei für den Kioskmodus. Wir ergänzen die Zeile um \code{-{}- -nocursor}.

\begin{figure}[H]
\begin{lstlisting}
sudo nano /lib/systemd/system/volumio-kiosk.service
# Original line
# ExecStart=/usr/bin/startx /etc/X11/Xsession /opt/volumiokiosk.sh
# Modified line
ExecStart=/usr/bin/startx /etc/X11/Xsession /opt/volumiokiosk.sh -- -nocursor
# Editor mit CTRL+X verlassen
\end{lstlisting}
\caption{Kiosk Modus}\label{fig:Kiosk Modus}
\end{figure}

\subsubsection{Bildschirmschoner}
Ab und zu wird das Display einfach \textit{abgeschaltet}. Das heißt, es wird vollständig schwarz. Um das zu verhindern, sind folgende Schritte notwendig:

\begin{figure}[H]
\begin{lstlisting}
sudo nano /opt/volumiokiosk.sh

# Originale Zeilen
# xset +dpms
# xset s blank
# xset 0 0 120

# Angepasste Zeilen
xset -dpms
xset s off
#xset 0 0 120
# Editor mit CTRL+X verlassen
\end{lstlisting}
\caption{Bildschirmschoner}\label{fig:Bildschirmschoner}
\end{figure}

\subsubsection{Zugriff von Windows}
Bei \vol{} ist von Haus aus bereits ein Samba installiert. Damit kann man problemlos auf den Speicher zugreifen. Allerdings bietet das Gerät verschiedene
Speicherorte an. Das könnte zu Verwirrungen sorgen. Deswegen sorgen wir dafür, dass lediglich auf Speicher, der über USB angeschlossen wird, zugegriffen werden
kann. Somit können wir von Windows aus ohne Rätselraten auf die SSD zugreifen.

\begin{figure}[H]
\begin{lstlisting}
sudo nano /etc/samba/smb.conf

# Originale Zeilen
[Internal Storage]
        comment = Boombox Internal Music Folder
        path = /data/INTERNAL
        read only = no
        guest ok = yes

[USB]
        comment = Boombox USB Music Folder
        path = /mnt/USB
        read only = no
        guest ok = yes

[NAS]
        comment = Boombox NAS Music Folder
        path = /mnt/NAS
        read only = no
        guest ok = yes

# Angepasste Zeilen
#[Internal Storage]
#        comment = Boombox Internal Music Folder
#        path = /data/INTERNAL
#        read only = no
#        guest ok = yes

[SSD]
        comment = Boombox SSD Music Folder
        path = /mnt/USB
        read only = no
        guest ok = yes

#[NAS]
#        comment = Boombox NAS Music Folder
#        path = /mnt/NAS
#        read only = no
#        guest ok = yes

# Editor mit CTRL+X verlassen
\end{lstlisting}
\caption{Share}\label{fig:Share}
\end{figure}

Unter Windows ist das Gerät dann unter dem Namen \code{\textbackslash{}\textbackslash{}boombox} im Windows Explorer erreichbar.

\jpaimg{./../images/win-bb.png}{Zugriff von Windows}

Zum Abschluss schalten wir den \textit{SSH} Zufriff wieder ab. Dazu rufen wir die entsprechende Seite auf. Siehe hierzu auch Kapitel \nameref{subsubsec:SSH}.

\clearpage{}
\phantomsection{}
\addcontentsline{toc}{section}{Abbildungsverzeichnis}
\listoffigures\thispagestyle{fancy}
\newpage

% \clearpage{}
% \phantomsection{}
% \addcontentsline{toc}{section}{Tabellenverzeichnis}
% \listoftables\thispagestyle{fancy}
% \newpage

\renewcommand{\indexname}{Stichwortverzeichnis}
\clearpage{}
\phantomsection{}
\addcontentsline{toc}{section}{Stichwortverzeichnis}
\printindex\thispagestyle{fancy}
\newpage

\end{document}
